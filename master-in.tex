\title{Rokkoチュートリアル}

\begin{document}

\lstset{language={sh},showspaces=false,rulecolor=\color[cmyk]{0, 0.29,0.84,0}}

\begin{frame}
  \titlepage
  \noindent {\footnotesize 本資料のソースは\url{https://github.com/cmsi/rokko-tutorial}にて公開中}
\end{frame}

\section*{Outline}
\begin{frame}
  \tableofcontents
\end{frame}

\section{チュートリアルの概要}

\begin{frame}{Rokkoチュートリアル スタッフ}
  \begin{itemize}
  \item 講師
    \begin{itemize}
    \item 藤堂眞治 (東大院理) \ \href{mailto:wistaria@phys.s.u-tokyo.ac.jp}{wistaria@phys.s.u-tokyo.ac.jp}
    \item 五十嵐 亮 (東大物性研) \ \href{mailto:rigarash@issp.u-tokyo.ac.jp}{rigarash@issp.u-tokyo.ac.jp}
    \end{itemize}
  \item 主催
    \begin{itemize}
    \item CMSI: 計算物質科学イニシアティブ \url{http://cms-initiative.jp/}
    \end{itemize}
  \end{itemize}
\end{frame}

\begin{frame}
  \frametitle{チュートリアルの流れ}
  \begin{itemize}
    %\setlength{\itemsep}{1em}
  \item 座学: 既存の固有値問題の解法・固有値ソルバ/線形計算ライブラリ
  \item 座学: Rokkoの概要と内部構造
  \item 実習: サンプルの実行
  \item 実習: アプリケーションからのRokkoの利用
  \item 座学: Rokkoのインストール
  \item 付録: ALPS/Baristaパッケージ
  \item 付録: MateriAppsとMateriApps LIVE!
  \end{itemize}
\end{frame}

\begin{frame}
  \frametitle{ネットワーク設定}
  \begin{itemize}
    %\setlength{\itemsep}{1em}
  \item LAN接続 (無線 or 有線)
  \item 実習用ワークステーションのアカウント登録
  \end{itemize}
\end{frame}

\section{固有値問題の解法・固有値ソルバ/線形計算ライブラリ}

\begin{frame}
  \frametitle{アウトライン}
  \begin{itemize}
    %\setlength{\itemsep}{1em}
  \item 行列の対角化
  \item 固有値問題の解法
  \item 既存の固有値ソルバ/線形計算ライブラリ
  \end{itemize}
\end{frame}

\section{Rokkoの概要と内部構造}

\begin{frame}
  \frametitle{アウトライン}
  \begin{itemize}
    %\setlength{\itemsep}{1em}
  \item 既存のライブラリの問題点
  \item Rokkoの概要
  \end{itemize}
\end{frame}

\section{サンプルの実行}
\section{アプリケーションからのRokkoの利用}
\section{Rokkoのインストール}
\section{ALPS/Baristaパッケージ}
\section{MateriAppsとMateriApps LIVE!}

\end{document}
