\title{Rokkoチュートリアル}

\begin{document}

\lstset{language={sh},showspaces=false,rulecolor=\color[cmyk]{0, 0.29,0.84,0}}

\begin{frame}
  \titlepage
  \noindent {\footnotesize 本資料のソースは\url{https://github.com/cmsi/rokko-tutorial}にて公開中}
\end{frame}

%% \section*{Outline}
%% \begin{frame}
%%   \tableofcontents
%% \end{frame}

\section{チュートリアルの概要}

\begin{frame}{Rokkoチュートリアル スタッフ}
  \begin{itemize}
  \item 講師
    \setlength{\itemsep}{1em}
    \begin{itemize}
    \item 藤堂眞治 (東大院理) \ \href{mailto:wistaria@phys.s.u-tokyo.ac.jp}{wistaria@phys.s.u-tokyo.ac.jp}
    \item 五十嵐 亮 (東大物性研) \ \href{mailto:rigarash@issp.u-tokyo.ac.jp}{rigarash@issp.u-tokyo.ac.jp}
    \item 本山裕一 (東大院工 $\Rightarrow $ 物性研) \ \href{mailto:yomichi@looper.u-tokyo.ac.jp}{yomichi@looper.t.u-tokyo.ac.jp}
    \end{itemize}
  \item 主催
    \begin{itemize}
    \item CMSI: 計算物質科学イニシアティブ \url{http://cms-initiative.jp/}
    \end{itemize}
  \end{itemize}
\end{frame}

\begin{frame}
  \frametitle{チュートリアルの流れ}
  \begin{itemize}
    %\setlength{\itemsep}{1em}
  \item 座学: 既存の固有値問題の解法・固有値ソルバ/線形計算ライブラリ
  \item 座学: Rokkoの概要と内部構造
  \item 実習: サンプルの実行
  \item 実習: アプリケーションからのRokkoの利用
  \item 座学: Rokkoのインストール
  \item 付録: ALPS/Baristaパッケージ
  \item 付録: MateriAppsとMateriApps LIVE!
  \end{itemize}
\end{frame}

\begin{frame}
  \frametitle{ネットワーク設定}
  \begin{itemize}
    \setlength{\itemsep}{1em}
  \item LAN接続 (無線 or 有線)
  \item 実習用ワークステーション(psi)のアカウント登録
  \end{itemize}
\end{frame}

\section{固有値問題の解法・固有値ソルバ/線形計算ライブラリ}

\begin{frame}
  \frametitle{アウトライン}
  \begin{itemize}
    %\setlength{\itemsep}{1em}
  \item 行列の対角化
  \item 固有値問題の解法
  \item 既存の固有値ソルバ/線形計算ライブラリ
  \end{itemize}
\end{frame}

\begin{frame}
  \frametitle{行列の対角化}
  \begin{itemize}
    \setlength{\itemsep}{1em}
  \item 行列の種類: 実対称行列、実非対称行列、エルミート行列、非エルミート行列
  \item 行列の表示: 密行列、CRS形式、MatFree形式 \\
    (それぞれ、TITPACKの「小規模」、「中規模」、「大規模」に対応)
  \item 必要な固有値: 全て、絶対値の大きな(小さな)順にいくつか、ある範囲内
  \item 固有ベクトル: 要/不要
  \end{itemize}
\end{frame}

\begin{frame}
  \frametitle{用語の定義}
  \begin{itemize}
    %\setlength{\itemsep}{1em}
  \item 固有値問題の解法(Eigenvalue algorithm) \\
    固有値問題を解くためのアルゴリズム
  \item 固有値ソルバ(Eigensolver, Eigenvalue problem solver) \\
    固有値解法の実装
  \item 固有値ソルバライブラリ(Eigensolver LIbrary) \\
    固有値ソルバのみを含むライブラリ
  \item 線形計算ライブラリ(Linear Algebra Library) \\
    固有値ソルバや他の線形計算ソルバの集合体
  \item 厳密対角化パッケージ(Exact diaognalization package) \\
    量子格子模型のハミルトニアンの固有値問題を扱うソフトウェア
  \end{itemize}
\end{frame}

\begin{frame}
  \frametitle{固有値問題の解法}
  \begin{itemize}
    \setlength{\itemsep}{1em}
  \item 三重対角行列に対する固有値問題の解法: 二分法, QR法, MR3, 分割統治法+QR法
  \item 密行列の直接対角化: Jacobi法
  \item 密行列の三重対角化: Householder法
  \item 疎行列の直接対角化: べき乗法, 逆べき乗法, レイリー商反復法, Jacobi-Davidson法, LOBPCG, Krylov-Schur法
  \item 疎行列の三重対角化: Lanczos法, Arnoldi法, リスタート付きLanczos法(RestartLanczos), Thick-restart Lanczos
  \item その他の方法: Sakurai-Sugiura法
  \end{itemize}
\end{frame}

\begin{frame}
  \frametitle{既存の固有値ソルバライブラリ}
  \begin{itemize}
    %\setlength{\itemsep}{1em}
  \item Anasazi: 反復法ソルバ中心。Krylov-Schur、Jacobi-Davidson、XXX-Davidson、LOBPCG、Implicit Riemann Trust Region Method
  \item EigenExa
  \item ELPA: Householder+分割統治法+QR
  \item IETL: ALPSに含まれる反復法ソルバ。Lanczos、他
  \end{itemize}
\end{frame}

\begin{frame}
  \frametitle{既存の線形計算ライブラリ}
  \begin{itemize}
    %\setlength{\itemsep}{1em}
  \item Apple VecLib: LAPACK
  \item ARPACK: Implicit Restarted Lanczos
  \item BLOPEX
  \item Eigen3: Householder+QR \\
    並列(ただし、プロセス数が平方数の場合しか使えない)
  \item Elemental: Householder+MR3
  \item Fujitsu SSLII: LAPACK, ScaLAPACK(の一部)、他(富士通独自)
  \item Intel MKL: LAPACK, ScaLAPACK
  \item Netlib LAPACK: LAPACKのリファレンス実装、Householder+QR、Householder+分割統治法+QR、Householder+二分法、Householder+MR3
  \item Netlib ScaLAPACK: Householder+QR、Householder+分割統治法+QR、Householder+二分法、Householder+MR3
  \item SLEPc: 反復法ソルバ中心。Krylov-Schur、Generalized Davidson、Jacobi-Davidson、Rayleigh Quotient Conjugate Gradient、Contour integral Sakurai-Sugiura、Power method、Subspace Itertation、Arnoldi (explicit restart)、Lanczos (explicit restart) \\
    ビルド時に逐次かMPI並列かを選ぶ必要あり
  \item Xabclib (ppOpen AT)
  \end{itemize}
\end{frame}

\begin{frame}
  \frametitle{おすすめ順}
  \begin{itemize}
    %\setlength{\itemsep}{1em}
  \item 密行列・逐次 \\
    LAPACK (のベンダー実装) $>$ Eigen3
  \item 密行列・MPI並列 \\
    EigenExa $>$ ELPA $>$ ScaLAPACK (のベンダー実装) $>$ Elemental
  \item 疎行列・逐次 \\
    Anasazi
  \item 疎行列・MPI並列 \\
    Anasazi $>$ SLEPc
  \end{itemize}
\end{frame}

\begin{frame}{EigenExa}
  \begin{center}
    \includegraphics[height=0.8\textheight]{figure/eigenexa.pdf}
  \end{center}
\end{frame}

\begin{frame}
  \frametitle{厳密対角化パッケージ}
  \begin{itemize}
    %\setlength{\itemsep}{1em}
  \item ALPS
  \item KOBEPACK
  \item SPINPACK
  \item TITPACK
  \end{itemize}
\end{frame}

\section{Rokkoの概要と内部構造}

\begin{frame}
  \frametitle{アウトライン}
  \begin{itemize}
    %\setlength{\itemsep}{1em}
  \item 既存のライブラリの問題点
  \item Rokkoの概要
  \end{itemize}
\end{frame}

\begin{frame}
  \frametitle{既存のライブラリの問題点}
  \begin{itemize}
    %\setlength{\itemsep}{1em}
  \item ソルバ毎に異なるデザイン
  \item インストール方法もライブラリ毎に異なる
  \item ドキュメントが不十分な場合も多い
  \item コンピュータのアーキテクチャ毎に異なるコンパイル・リンクオプションが必要
  \item C++/C/Fortran相互リンクの問題
  \item ライブラリ間の依存関係が複雑
  \item 実際に試す前に大まかな性能比較が欲しい
  \end{itemize}
\end{frame}

\begin{frame}
  \frametitle{ライブラリ間の依存関係(一部)}
  \begin{center}
    \includegraphics[height=0.8\textheight]{figure/library-dependence.pdf}
  \end{center}
\end{frame}

\begin{frame}
  \frametitle{Rokkoの開発者}
  \begin{itemize}
    \setlength{\itemsep}{1em}
  \item 坂下達哉 (東大物性研) \ \href{mailto:t-sakashita@issp.u-tokyo.ac.jp}{t-sakashita@issp.u-tokyo.ac.jp}
  \item 本山裕一 (東大院工 $\Rightarrow $ 物性研) \ \href{mailto:yomichi@looper.u-tokyo.ac.jp}{yomichi@looper.t.u-tokyo.ac.jp}
  \item 五十嵐 亮 (東大物性研) \ \href{mailto:rigarash@issp.u-tokyo.ac.jp}{rigarash@issp.u-tokyo.ac.jp}
  \item 大久保 毅 (東大物性研) \ \href{mailto:t-okubo@issp.u-tokyo.ac.jp}{t-okubo@issp.u-tokyo.ac.jp}
  \item 藤堂眞治 (東大院理/東大物性研) \ \href{mailto:wistaria@phys.s.u-tokyo.ac.jp}{wistaria@phys.s.u-tokyo.ac.jp}
  \end{itemize}
\end{frame}

\begin{frame}
  \frametitle{Rokkoの概要}
  \begin{itemize}
    %\setlength{\itemsep}{1em}
  \item 使用言語
    \begin{itemize}
      %\setlength{\itemsep}{1em}
    \item コア部分: C++
    \item 言語バインディング: C, Fortran90
    \item ベンチマークスクリプト: Python
    \end{itemize}
  \item ライセンス
    \begin{itemize}
      %\setlength{\itemsep}{1em}
    \item Boostライセンス (ほぼ自由に使える)
    \end{itemize}
  \item ソースコード
    \begin{itemize}
      %\setlength{\itemsep}{1em}
    \item GitHubで公開
    \item \url{https://github.com/t-sakashita/rokko}
    \end{itemize}
  \end{itemize}
\end{frame}

\begin{frame}
  \frametitle{Rokkoの設計思想}
  \begin{itemize}
    %\setlength{\itemsep}{1em}
  \item 共通の行列クラス
  \item 実行時にソルバを選択可能
  \end{itemize}
\end{frame}

\begin{frame}
  \frametitle{Rokkoの全体像}
  \begin{itemize}
    %\setlength{\itemsep}{1em}
  \item 固有値ソルバ/線形演算ライブラリのインストールスクリプト
  \item 共通基本クラス(分散行列、プロセスグリッド他)
  \item 固有値ソルバラッパー(C++)
  \item 固有値ソルバ・ファクトリ(C++)
  \item C/Fortranラッパー
  \item テスト・サンプルプログラム
  \item ベンチマークスクリプト
  \end{itemize}
\end{frame}

\section{並列ソルバの基本概念}

\begin{frame}
  \frametitle{プロセスグリッド}
  \begin{itemize}
    %\setlength{\itemsep}{1em}
  \item 
  \end{itemize}
\end{frame}

\section{サンプルの実行}
\section{アプリケーションからのRokkoの利用}
\section{Rokkoのインストール}
\section{ALPS/Baristaパッケージ}
\section{MateriAppsとMateriApps LIVE!}

\begin{frame}
  \frametitle{テスト}
  \begin{itemize}
    %\setlength{\itemsep}{1em}
  \item テスト
  \end{itemize}
\end{frame}

\end{document}
